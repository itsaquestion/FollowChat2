% Options for packages loaded elsewhere
\PassOptionsToPackage{unicode}{hyperref}
\PassOptionsToPackage{hyphens}{url}
\PassOptionsToPackage{dvipsnames,svgnames,x11names}{xcolor}
%
\documentclass[
  letterpaper,
  DIV=11,
  numbers=noendperiod]{scrreprt}

\usepackage{amsmath,amssymb}
\usepackage{iftex}
\ifPDFTeX
  \usepackage[T1]{fontenc}
  \usepackage[utf8]{inputenc}
  \usepackage{textcomp} % provide euro and other symbols
\else % if luatex or xetex
  \usepackage{unicode-math}
  \defaultfontfeatures{Scale=MatchLowercase}
  \defaultfontfeatures[\rmfamily]{Ligatures=TeX,Scale=1}
\fi
\usepackage{lmodern}
\ifPDFTeX\else  
    % xetex/luatex font selection
\fi
% Use upquote if available, for straight quotes in verbatim environments
\IfFileExists{upquote.sty}{\usepackage{upquote}}{}
\IfFileExists{microtype.sty}{% use microtype if available
  \usepackage[]{microtype}
  \UseMicrotypeSet[protrusion]{basicmath} % disable protrusion for tt fonts
}{}
\makeatletter
\@ifundefined{KOMAClassName}{% if non-KOMA class
  \IfFileExists{parskip.sty}{%
    \usepackage{parskip}
  }{% else
    \setlength{\parindent}{0pt}
    \setlength{\parskip}{6pt plus 2pt minus 1pt}}
}{% if KOMA class
  \KOMAoptions{parskip=half}}
\makeatother
\usepackage{xcolor}
\setlength{\emergencystretch}{3em} % prevent overfull lines
\setcounter{secnumdepth}{5}
% Make \paragraph and \subparagraph free-standing
\ifx\paragraph\undefined\else
  \let\oldparagraph\paragraph
  \renewcommand{\paragraph}[1]{\oldparagraph{#1}\mbox{}}
\fi
\ifx\subparagraph\undefined\else
  \let\oldsubparagraph\subparagraph
  \renewcommand{\subparagraph}[1]{\oldsubparagraph{#1}\mbox{}}
\fi


\providecommand{\tightlist}{%
  \setlength{\itemsep}{0pt}\setlength{\parskip}{0pt}}\usepackage{longtable,booktabs,array}
\usepackage{calc} % for calculating minipage widths
% Correct order of tables after \paragraph or \subparagraph
\usepackage{etoolbox}
\makeatletter
\patchcmd\longtable{\par}{\if@noskipsec\mbox{}\fi\par}{}{}
\makeatother
% Allow footnotes in longtable head/foot
\IfFileExists{footnotehyper.sty}{\usepackage{footnotehyper}}{\usepackage{footnote}}
\makesavenoteenv{longtable}
\usepackage{graphicx}
\makeatletter
\def\maxwidth{\ifdim\Gin@nat@width>\linewidth\linewidth\else\Gin@nat@width\fi}
\def\maxheight{\ifdim\Gin@nat@height>\textheight\textheight\else\Gin@nat@height\fi}
\makeatother
% Scale images if necessary, so that they will not overflow the page
% margins by default, and it is still possible to overwrite the defaults
% using explicit options in \includegraphics[width, height, ...]{}
\setkeys{Gin}{width=\maxwidth,height=\maxheight,keepaspectratio}
% Set default figure placement to htbp
\makeatletter
\def\fps@figure{htbp}
\makeatother

\KOMAoption{captions}{tableheading}
\makeatletter
\makeatother
\makeatletter
\@ifpackageloaded{bookmark}{}{\usepackage{bookmark}}
\makeatother
\makeatletter
\@ifpackageloaded{caption}{}{\usepackage{caption}}
\AtBeginDocument{%
\ifdefined\contentsname
  \renewcommand*\contentsname{Table of contents}
\else
  \newcommand\contentsname{Table of contents}
\fi
\ifdefined\listfigurename
  \renewcommand*\listfigurename{List of Figures}
\else
  \newcommand\listfigurename{List of Figures}
\fi
\ifdefined\listtablename
  \renewcommand*\listtablename{List of Tables}
\else
  \newcommand\listtablename{List of Tables}
\fi
\ifdefined\figurename
  \renewcommand*\figurename{Figure}
\else
  \newcommand\figurename{Figure}
\fi
\ifdefined\tablename
  \renewcommand*\tablename{Table}
\else
  \newcommand\tablename{Table}
\fi
}
\@ifpackageloaded{float}{}{\usepackage{float}}
\floatstyle{ruled}
\@ifundefined{c@chapter}{\newfloat{codelisting}{h}{lop}}{\newfloat{codelisting}{h}{lop}[chapter]}
\floatname{codelisting}{Listing}
\newcommand*\listoflistings{\listof{codelisting}{List of Listings}}
\makeatother
\makeatletter
\@ifpackageloaded{caption}{}{\usepackage{caption}}
\@ifpackageloaded{subcaption}{}{\usepackage{subcaption}}
\makeatother
\makeatletter
\@ifpackageloaded{tcolorbox}{}{\usepackage[skins,breakable]{tcolorbox}}
\makeatother
\makeatletter
\@ifundefined{shadecolor}{\definecolor{shadecolor}{rgb}{.97, .97, .97}}
\makeatother
\makeatletter
\makeatother
\makeatletter
\makeatother
\ifLuaTeX
  \usepackage{selnolig}  % disable illegal ligatures
\fi
\IfFileExists{bookmark.sty}{\usepackage{bookmark}}{\usepackage{hyperref}}
\IfFileExists{xurl.sty}{\usepackage{xurl}}{} % add URL line breaks if available
\urlstyle{same} % disable monospaced font for URLs
\hypersetup{
  pdftitle={News},
  pdfauthor={lee},
  colorlinks=true,
  linkcolor={blue},
  filecolor={Maroon},
  citecolor={Blue},
  urlcolor={Blue},
  pdfcreator={LaTeX via pandoc}}

\title{News}
\author{lee}
\date{2023-12-02}

\begin{document}
\maketitle
\ifdefined\Shaded\renewenvironment{Shaded}{\begin{tcolorbox}[breakable, boxrule=0pt, interior hidden, sharp corners, frame hidden, borderline west={3pt}{0pt}{shadecolor}, enhanced]}{\end{tcolorbox}}\fi

\renewcommand*\contentsname{Table of contents}
{
\hypersetup{linkcolor=}
\setcounter{tocdepth}{2}
\tableofcontents
}
\bookmarksetup{startatroot}

\hypertarget{preface}{%
\chapter*{Preface}\label{preface}}
\addcontentsline{toc}{chapter}{Preface}

\markboth{Preface}{Preface}

\bookmarksetup{startatroot}

\hypertarget{us-to-invest-3-billion-in-green-climate-fund}{%
\chapter{US to Invest \$3 Billion in Green Climate
Fund}\label{us-to-invest-3-billion-in-green-climate-fund}}

\hypertarget{professional-english}{%
\section{Professional English}\label{professional-english}}

The United States has announced plans to invest \$3 billion into the
Green Climate Fund (GCF), according to sources familiar with the matter.
The GCF is a global fund that supports developing countries in their
efforts to combat climate change and adapt to its impacts. The
investment is part of the US government's commitment to addressing
climate change and promoting sustainable development. The funds will be
used to finance projects and initiatives aimed at reducing greenhouse
gas emissions, promoting renewable energy, and supporting climate
resilience in vulnerable communities. This announcement comes as
countries around the world are ramping up their efforts to combat
climate change and transition to a low-carbon economy.

\hypertarget{simplified-english}{%
\section{Simplified English}\label{simplified-english}}

The United States has announced that it will invest \$3 billion in the
Green Climate Fund (GCF), a global fund that helps developing countries
fight climate change and adapt to its effects. This investment is part
of the US government's commitment to addressing climate change and
promoting sustainable development. The money will be used to support
projects that aim to reduce greenhouse gas emissions, promote renewable
energy, and help vulnerable communities cope with the impacts of climate
change. This announcement comes at a time when countries worldwide are
increasing their efforts to combat climate change and transition to a
low-carbon economy.

\hypertarget{spoken-english}{%
\section{Spoken English}\label{spoken-english}}

The United States is putting \$3 billion into the Green Climate Fund, a
global fund that helps developing countries deal with climate change.
The money will be used to support projects that reduce greenhouse gas
emissions, promote renewable energy, and help communities that are most
affected by climate change. This announcement is part of the US
government's commitment to addressing climate change and promoting
sustainable development. It comes at a time when countries all over the
world are working harder to fight climate change and move towards a
cleaner, more sustainable future.

\hypertarget{spoken-english-with-pause-tag}{%
\section{Spoken English with pause
tag}\label{spoken-english-with-pause-tag}}

The United States is putting \$3 billion into the Green Climate Fund, a
global fund that helps developing countries deal with climate change.
The money will be used to support projects that reduce greenhouse gas
emissions, promote renewable energy, and help communities that are most
affected by climate change. This announcement is part of the US
government's commitment to addressing climate change and promoting
sustainable development. It comes at a time when countries all over the
world are working harder to fight climate change and move towards a
cleaner, more sustainable future.

\hypertarget{vocabulary}{%
\section{Vocabulary}\label{vocabulary}}

\begin{itemize}
\tightlist
\item
  Green Climate Fund /ɡriːn ˈklaɪmət fʌnd/: 绿色气候基金
\item
  greenhouse gas emissions /ˈɡriːnhaʊs ɡæs ɪˈmɪʃənz/: 温室气体排放
\item
  renewable energy /rɪˈnjuːəbəl ˈɛnərdʒi/: 可再生能源
\item
  vulnerable communities /ˈvʌlnərəbl kəˈmjuːnətiz/: 脆弱社区
\item
  sustainable development /səˈsteɪnəbl dɪˈvɛləpmənt/: 可持续发展
\item
  climate resilience /ˈklaɪmət rɪˈzɪliəns/: 气候适应能力
\item
  combat climate change /ˈkɒmbæt ˈklaɪmət tʃeɪndʒ/: 应对气候变化
\item
  low-carbon economy /ləʊ ˈkɑːbən ɪˈkɒnəmi/: 低碳经济
\end{itemize}

\bookmarksetup{startatroot}

\hypertarget{us-to-invest-3-billion-in-green-climate-fund-1}{%
\chapter{US to Invest \$3 Billion in Green Climate
Fund}\label{us-to-invest-3-billion-in-green-climate-fund-1}}

\hypertarget{professional-english-1}{%
\section{Professional English}\label{professional-english-1}}

The United States has announced plans to invest \$3 billion into the
Green Climate Fund, according to sources familiar with the matter. The
move is seen as a significant step towards addressing climate change and
promoting sustainable finance. The Green Climate Fund, established in
2010, aims to support developing countries in their efforts to mitigate
and adapt to climate change. The fund provides financial assistance for
projects that reduce greenhouse gas emissions and enhance resilience to
climate impacts. The US investment is expected to boost the fund's
capacity to support climate-related initiatives around the world.

\hypertarget{simplified-english-1}{%
\section{Simplified English}\label{simplified-english-1}}

The United States has revealed its intention to allocate \$3 billion to
the Green Climate Fund, as reported by sources who are knowledgeable
about the matter. This decision is considered a significant move towards
tackling climate change and promoting sustainable finance. The Green
Climate Fund, which was established in 2010, aims to assist developing
nations in their endeavors to mitigate and adapt to climate change. The
fund provides financial aid for projects that aim to reduce greenhouse
gas emissions and enhance resilience to climate-related impacts. The
investment from the US is expected to enhance the fund's capacity to
support climate-related initiatives globally.

\hypertarget{spoken-english-1}{%
\section{Spoken English}\label{spoken-english-1}}

The United States has just announced that it will be putting \$3 billion
into the Green Climate Fund. This is a fund that was set up back in 2010
to help developing countries deal with climate change. It gives money to
projects that reduce the amount of greenhouse gases that are being
released into the atmosphere and helps countries become more resilient
to the impacts of climate change. This investment from the US is going
to make a big difference in the fund's ability to support projects all
around the world.

\hypertarget{spoken-english-with-pause-tag-1}{%
\section{Spoken English with pause
tag}\label{spoken-english-with-pause-tag-1}}

The United States has just announced that it will be putting \$3 billion
into the Green Climate Fund. This is a fund that was set up back in 2010
to help developing countries deal with climate change. It gives money to
projects that reduce the amount of greenhouse gases that are being
released into the atmosphere and helps countries become more resilient
to the impacts of climate change. This investment from the US is going
to make a big difference in the fund's ability to support projects all
around the world.

\hypertarget{vocabulary-1}{%
\section{Vocabulary}\label{vocabulary-1}}

\begin{itemize}
\tightlist
\item
  Green Climate Fund /ɡriːn ˈklaɪmət fʌnd/: 绿色气候基金
\item
  mitigate /ˈmɪtɪɡeɪt/: 减轻,缓解
\item
  adapt /əˈdæpt/: 适应
\item
  greenhouse gas emissions /ˈɡriːnhaʊs ɡæs ɪˈmɪʃənz/: 温室气体排放
\item
  resilience /rɪˈzɪliəns/: 韧性,适应力
\item
  climate-related impacts /ˈklaɪmət rɪˌleɪtɪd ˈɪmpækts/:
  与气候相关的影响
\item
  allocate /ˈæləkeɪt/: 分配,拨款
\item
  sustainable finance /səˈsteɪnəbl ˈfaɪnæns/: 可持续金融
\item
  significant move /sɪɡˈnɪfɪkənt muːv/: 重要举措
\item
  tackle /ˈtækəl/: 应对,解决
\item
  enhance /ɪnˈhæns/: 增强
\item
  capacity /kəˈpæsəti/: 能力,容量
\item
  globally /ˈɡloʊbəli/: 全球范围内
\end{itemize}

\bookmarksetup{startatroot}

\hypertarget{us-to-invest-3-billion-in-green-climate-fund-2}{%
\chapter{US to Invest \$3 Billion in Green Climate
Fund}\label{us-to-invest-3-billion-in-green-climate-fund-2}}

\hypertarget{professional-english-2}{%
\section{Professional English}\label{professional-english-2}}

The United States has announced plans to invest \$3 billion in the Green
Climate Fund, according to sources familiar with the matter. The move is
part of the country's efforts to combat climate change and promote
sustainable finance. The Green Climate Fund, established in 2010, aims
to support developing countries in their transition to low-carbon and
climate-resilient economies. The fund provides financial assistance for
projects that reduce greenhouse gas emissions and help vulnerable
communities adapt to the impacts of climate change. The US investment is
expected to contribute to global efforts to address climate change and
promote sustainable development.

\hypertarget{simplified-english-2}{%
\section{Simplified English}\label{simplified-english-2}}

The United States has announced that it will invest \$3 billion in the
Green Climate Fund. This investment is aimed at fighting climate change
and promoting sustainable finance. The Green Climate Fund was created in
2010 to help developing countries transition to low-carbon and
climate-resilient economies. It provides financial support for projects
that reduce greenhouse gas emissions and help communities that are
vulnerable to the effects of climate change. The US investment is
expected to make a significant contribution to global efforts to address
climate change and promote sustainable development.

\hypertarget{spoken-english-2}{%
\section{Spoken English}\label{spoken-english-2}}

The United States just announced that it's going to put \$3 billion into
the Green Climate Fund. This is part of their plan to fight climate
change and support sustainable finance. The Green Climate Fund was set
up in 2010 to help developing countries switch to low-carbon and
climate-resilient economies. It gives money to projects that reduce
greenhouse gas emissions and help communities that are at risk from
climate change. The US investment is a big deal and will really help in
the global fight against climate change and efforts to promote
sustainable development.

\hypertarget{spoken-english-with-pause-tag-2}{%
\section{Spoken English with pause
tag}\label{spoken-english-with-pause-tag-2}}

The United States just announced that it's going to put \$3 billion into
the Green Climate Fund. This is part of their plan to fight climate
change and support sustainable finance. The Green Climate Fund was set
up in 2010 to help developing countries switch to low-carbon and
climate-resilient economies. It gives money to projects that reduce
greenhouse gas emissions and help communities that are at risk from
climate change. The US investment is a big deal and will really help in
the global fight against climate change and efforts to promote
sustainable development.

\hypertarget{vocabulary-2}{%
\section{Vocabulary}\label{vocabulary-2}}

\begin{itemize}
\tightlist
\item
  Green Climate Fund /ɡriːn ˈklaɪmət fʌnd/: 绿色气候基金
\item
  combat /ˈkɑːmbæt/: 战斗,对抗
\item
  sustainable finance /səˈsteɪnəbəl ˈfaɪnæns/: 可持续金融
\item
  low-carbon /ˌloʊ ˈkɑːrbən/: 低碳
\item
  climate-resilient /ˈklaɪmət rɪˈzɪliənt/: 抗气候变化的
\item
  greenhouse gas emissions /ˈɡriːnhaʊs ɡæs ɪˈmɪʃənz/: 温室气体排放
\item
  vulnerable /ˈvʌlnərəbl/: 脆弱的
\item
  sustainable development /səˈsteɪnəbəl dɪˈvɛləpmənt/: 可持续发展
\end{itemize}

\bookmarksetup{startatroot}

\hypertarget{us-to-invest-3-billion-in-green-climate-fund-3}{%
\chapter{US to Invest \$3 Billion in Green Climate
Fund}\label{us-to-invest-3-billion-in-green-climate-fund-3}}

\hypertarget{professional-english-3}{%
\section{Professional English}\label{professional-english-3}}

The United States has announced plans to invest \$3 billion into the
Green Climate Fund, according to sources familiar with the matter. The
move is seen as a significant step towards addressing climate change and
promoting sustainable finance. The Green Climate Fund, established in
2010, aims to support developing countries in their efforts to mitigate
and adapt to climate change. The fund provides financial assistance for
projects that reduce greenhouse gas emissions and enhance resilience to
climate impacts. The US investment is expected to boost international
efforts to combat climate change and encourage other countries to
contribute to the fund.

\hypertarget{simplified-english-3}{%
\section{Simplified English}\label{simplified-english-3}}

The United States has announced that it will invest \$3 billion in the
Green Climate Fund, as reported by sources familiar with the matter.
This investment is considered a significant move towards addressing
climate change and promoting sustainable finance. The Green Climate
Fund, which was established in 2010, aims to assist developing countries
in their efforts to tackle and adapt to climate change. The fund
provides financial support for projects that aim to reduce greenhouse
gas emissions and enhance resilience to climate impacts. The US
investment is expected to strengthen global initiatives in combating
climate change and encourage other countries to contribute to the fund.

\hypertarget{spoken-english-3}{%
\section{Spoken English}\label{spoken-english-3}}

The United States has just announced that it will invest a whopping \$3
billion in the Green Climate Fund. This fund, established back in 2010,
is all about helping developing countries deal with climate change. It
provides money for projects that reduce greenhouse gas emissions and
make communities more resilient to the impacts of climate change. The US
investment is a big deal because it shows that the country is serious
about taking action on climate change. It's also hoped that this move
will encourage other countries to contribute to the fund and work
together to combat this global challenge.

\hypertarget{spoken-english-with-pause-tag-3}{%
\section{Spoken English with pause
tag}\label{spoken-english-with-pause-tag-3}}

The United States has just announced that it will invest a whopping \$3
billion in the Green Climate Fund. This fund, established back in 2010,
is all about helping developing countries deal with climate change. It
provides money for projects that reduce greenhouse gas emissions and
make communities more resilient to the impacts of climate change. The US
investment is a big deal because it shows that the country is serious
about taking action on climate change. It's also hoped that this move
will encourage other countries to contribute to the fund and work
together to combat this global challenge.

\hypertarget{vocabulary-3}{%
\section{Vocabulary}\label{vocabulary-3}}

\begin{itemize}
\tightlist
\item
  Green Climate Fund /ɡriːn ˈklaɪmət fʌnd/: 绿色气候基金
\item
  sustainable finance /səˈsteɪnəbəl ˈfaɪnæns/: 可持续金融
\item
  mitigate /ˈmɪtɪɡeɪt/: 缓解
\item
  adapt to /əˈdæpt tuː/: 适应
\item
  greenhouse gas emissions /ˈɡriːnhaʊs ɡæs ɪˈmɪʃənz/: 温室气体排放
\item
  enhance resilience /ɪnˈhæns rɪˈzɪliəns/: 增强韧性
\item
  combat /ˈkɑːmbæt/: 打击,应对
\end{itemize}

\bookmarksetup{startatroot}

\hypertarget{mossad-team-meets-with-qatar-to-discuss-gaza-truce-restart}{%
\chapter{Mossad Team Meets with Qatar to Discuss Gaza Truce
Restart}\label{mossad-team-meets-with-qatar-to-discuss-gaza-truce-restart}}

\hypertarget{professional-english-4}{%
\section{Professional English}\label{professional-english-4}}

A team from Israel's intelligence agency, Mossad, reportedly met with
officials from Qatar to discuss the possibility of restarting the truce
in Gaza. According to a source, the meeting took place in Doha and
focused on finding a way to restore calm in the region. The truce, which
was brokered by Egypt, has been in place since May, but has been
repeatedly violated by both Israel and Hamas. The talks between Mossad
and Qatar come as tensions continue to escalate in Gaza, with recent
clashes between Israeli forces and Palestinian protesters resulting in
multiple casualties. The outcome of the meeting remains unclear, but it
is hoped that a renewed truce agreement can be reached to prevent
further violence in the region.

\hypertarget{simplified-english-4}{%
\section{Simplified English}\label{simplified-english-4}}

A team from Israel's intelligence agency, Mossad, recently had a meeting
with officials from Qatar to discuss the possibility of restarting the
truce in Gaza. The meeting took place in Doha and focused on finding a
way to bring back peace in the region. The truce, which was arranged by
Egypt, has been in effect since May, but has been repeatedly broken by
both Israel and Hamas. The talks between Mossad and Qatar come at a time
when tensions are rising in Gaza, with recent clashes between Israeli
forces and Palestinian protesters resulting in many people getting hurt.
It is not clear what the outcome of the meeting will be, but it is hoped
that a new truce agreement can be reached to stop more violence from
happening in the region.

\hypertarget{spoken-english-4}{%
\section{Spoken English}\label{spoken-english-4}}

A team from Israel's intelligence agency, Mossad, reportedly met with
officials from Qatar to discuss the possibility of restarting the truce
in Gaza. The meeting happened in Doha and focused on finding a way to
restore calm in the region. The truce, which was brokered by Egypt, has
been in place since May, but has been repeatedly violated by both Israel
and Hamas. The talks between Mossad and Qatar come as tensions continue
to escalate in Gaza, with recent clashes between Israeli forces and
Palestinian protesters resulting in multiple casualties. The outcome of
the meeting remains unclear, but it is hoped that a renewed truce
agreement can be reached to prevent further violence in the region.

\hypertarget{spoken-english-with-pause-tag-4}{%
\section{Spoken English with pause
tag}\label{spoken-english-with-pause-tag-4}}

A team from Israel's intelligence agency, Mossad, reportedly met with
officials from Qatar to discuss the possibility of restarting the truce
in Gaza. The meeting happened in Doha and focused on finding a way to
restore calm in the region. The truce, which was brokered by Egypt, has
been in place since May but has been repeatedly violated by both Israel
and Hamas. The talks between Mossad and Qatar come as tensions continue
to escalate in Gaza with recent clashes between Israeli forces and
Palestinian protesters resulting in multiple casualties. The outcome of
the meeting remains unclear but it is hoped that a renewed truce
agreement can be reached to prevent further violence in the region.

\hypertarget{vocabulary-4}{%
\section{Vocabulary}\label{vocabulary-4}}

\begin{itemize}
\tightlist
\item
  Mossad /ˈmɒsæd/ : 以色列情报机构
\item
  truce /truːs/ : 停火协议
\item
  brokered /ˈbroʊkərd/ : 斡旋
\item
  violated /ˈvaɪəleɪtɪd/ : 违反
\item
  escalate /ˈɛskəleɪt/ : 升级
\item
  clashes /ˈklæʃɪz/ : 冲突
\item
  casualties /ˈkæʒuəltiz/ : 伤亡人员
\item
  outcome /ˈaʊtkʌm/ : 结果
\item
  renewed /rɪˈnjuːd/ : 更新
\item
  prevent /prɪˈvɛnt/ : 阻止
\end{itemize}

\bookmarksetup{startatroot}

\hypertarget{mossad-team-in-qatar-to-discuss-gaza-truce-restart}{%
\chapter{Mossad Team in Qatar to Discuss Gaza Truce
Restart}\label{mossad-team-in-qatar-to-discuss-gaza-truce-restart}}

\hypertarget{professional-english-5}{%
\section{Professional English}\label{professional-english-5}}

According to a source, a team from the Israeli intelligence agency,
Mossad, has traveled to Qatar to discuss the possibility of restarting
the truce in Gaza. The team's visit comes after recent escalations in
violence between Israel and Hamas. The discussions are aimed at finding
a way to de-escalate the situation and prevent further bloodshed. The
truce, which was brokered by Egypt in May, has been fragile, with
sporadic clashes and rocket attacks occurring since then. The Mossad
team's visit to Qatar indicates a diplomatic effort to find a peaceful
resolution to the ongoing conflict.

\hypertarget{simplified-english-5}{%
\section{Simplified English}\label{simplified-english-5}}

A team from the Israeli intelligence agency, Mossad, has gone to Qatar
to talk about restarting the truce in Gaza, according to a source. This
comes after recent increases in violence between Israel and Hamas. The
talks are trying to find a way to make the situation less tense and stop
more fighting. The truce, which Egypt helped make in May, has been weak,
with some fighting and rocket attacks happening since then. The Mossad
team going to Qatar shows that they are trying to find a peaceful way to
end the fighting.

\hypertarget{spoken-english-5}{%
\section{Spoken English}\label{spoken-english-5}}

A team from the Israeli intelligence agency, Mossad, has traveled to
Qatar to discuss the possibility of restarting the truce in Gaza,
according to a source. This comes after recent escalations in violence
between Israel and Hamas. The discussions are aimed at finding a way to
de-escalate the situation and prevent further bloodshed. The truce,
which was brokered by Egypt in May, has been fragile, with sporadic
clashes and rocket attacks occurring since then. The Mossad team's visit
to Qatar indicates a diplomatic effort to find a peaceful resolution to
the ongoing conflict.

\hypertarget{spoken-english-with-pause-tag-5}{%
\section{Spoken English with pause
tag}\label{spoken-english-with-pause-tag-5}}

A team from the Israeli intelligence agency, Mossad, has traveled to
Qatar to discuss the possibility of restarting the truce in Gaza,
according to a source. This comes after recent escalations in violence
between Israel and Hamas. The discussions are aimed at finding a way to
de-escalate the situation and prevent further bloodshed. The truce,
which was brokered by Egypt in May, has been fragile with sporadic
clashes and rocket attacks occurring since then. The Mossad team's visit
to Qatar indicates a diplomatic effort to find a peaceful resolution to
the ongoing conflict.

\hypertarget{vocabulary-5}{%
\section{Vocabulary}\label{vocabulary-5}}

\begin{itemize}
\tightlist
\item
  Mossad /ˈmɒsæd/: 以色列情报机构
\item
  truce /truːs/: 停火协议
\item
  de-escalate /diːˈeskəleɪt/: 缓和
\item
  bloodshed /ˈblʌdʃed/: 流血事件
\item
  brokered /ˈbroʊkərd/: 斡旋
\item
  sporadic /spəˈrædɪk/: 零星的
\item
  clashes /klæʃɪz/: 冲突
\item
  rocket attacks /ˈrɒkɪt əˈtæks/: 火箭袭击
\item
  diplomatic effort /ˌdɪpləˈmætɪk ˈɛfərt/: 外交努力
\item
  ongoing /ˈɒnˌɡoʊɪŋ/: 持续的
\end{itemize}

\bookmarksetup{startatroot}

\hypertarget{us-to-invest-3-billion-in-green-climate-fund-4}{%
\chapter{US to Invest \$3 Billion in Green Climate
Fund}\label{us-to-invest-3-billion-in-green-climate-fund-4}}

\hypertarget{professional-english-6}{%
\section{Professional English}\label{professional-english-6}}

The United States has announced plans to invest \$3 billion into the
Green Climate Fund, according to sources familiar with the matter. The
investment is expected to be announced at the upcoming COP26 climate
summit in Glasgow. The Green Climate Fund, established in 2010, aims to
support developing countries in their efforts to combat climate change
and adapt to its impacts. The fund provides financial assistance for
projects related to renewable energy, energy efficiency, and climate
resilience. The US investment is seen as a significant step towards
fulfilling the country's commitment to climate action and supporting
global efforts to address the climate crisis.

\hypertarget{simplified-english-6}{%
\section{Simplified English}\label{simplified-english-6}}

The United States has announced that it will invest \$3 billion in the
Green Climate Fund. This investment will be revealed at the COP26
climate summit in Glasgow. The Green Climate Fund was created in 2010 to
help developing countries fight climate change and adapt to its effects.
The fund provides money for projects that focus on renewable energy,
energy efficiency, and climate resilience. The US investment is an
important move towards fulfilling the country's promise to take action
on climate change and support worldwide efforts to tackle the climate
crisis.

\hypertarget{spoken-english-6}{%
\section{Spoken English}\label{spoken-english-6}}

The United States is putting \$3 billion into the Green Climate Fund.
They're going to announce this investment at the COP26 climate summit in
Glasgow. The Green Climate Fund was set up in 2010 to help developing
countries deal with climate change and adapt to its impacts. It gives
money to projects that are about renewable energy, energy efficiency,
and being able to handle the effects of climate change. This investment
from the US is a big step towards keeping their promise to do something
about climate change and helping the whole world deal with the climate
crisis.

\hypertarget{spoken-english-with-pause-tag-6}{%
\section{Spoken English with pause
tag}\label{spoken-english-with-pause-tag-6}}

The United States is putting \$3 billion into the Green Climate Fund.
They're going to announce this investment at the COP26 climate summit in
Glasgow. The Green Climate Fund was set up in 2010 to help developing
countries deal with climate change and adapt to its impacts. It gives
money to projects that are about renewable energy , energy efficiency ,
and being able to handle the effects of climate change. This investment
from the US is a big step towards keeping their promise to do something
about climate change and helping the whole world deal with the climate
crisis.

\hypertarget{vocabulary-6}{%
\section{Vocabulary}\label{vocabulary-6}}

\begin{itemize}
\tightlist
\item
  Green Climate Fund /ɡriːn ˈklaɪmət fʌnd/: 绿色气候基金
\item
  COP26 /ˈkɒp ˈtwɛnti ˈsɪks/: 第26届联合国气候变化大会
\item
  renewable energy /rɪˈnjuːəbəl ˈɛnərdʒi/: 可再生能源
\item
  energy efficiency /ˈɛnərdʒi ɪˈfɪʃənsi/: 能源效率
\item
  climate resilience /ˈklaɪmət rɪˈzɪliəns/: 气候适应能力
\item
  climate crisis /ˈklaɪmət ˈkraɪsɪs/: 气候危机
\end{itemize}

\bookmarksetup{startatroot}

\hypertarget{us-to-invest-3-billion-in-green-climate-fund-5}{%
\chapter{US to Invest \$3 Billion in Green Climate
Fund}\label{us-to-invest-3-billion-in-green-climate-fund-5}}

\hypertarget{professional-english-7}{%
\section{Professional English}\label{professional-english-7}}

The United States has announced plans to invest \$3 billion in the Green
Climate Fund (GCF), according to sources familiar with the matter. The
GCF is a financial mechanism established by the United Nations Framework
Convention on Climate Change to assist developing countries in their
efforts to combat climate change. The investment is part of the US
government's commitment to addressing climate change and supporting
global efforts to transition to a low-carbon economy. The funds will be
used to support projects and initiatives aimed at reducing greenhouse
gas emissions, promoting renewable energy, and building climate
resilience in developing countries. This announcement comes as world
leaders gather for the COP26 climate summit in Glasgow, Scotland, where
discussions on climate finance and the need for increased funding for
climate action are taking place.

\hypertarget{simplified-english-7}{%
\section{Simplified English}\label{simplified-english-7}}

The United States has announced that it will invest \$3 billion in the
Green Climate Fund (GCF). The GCF is a financial mechanism created by
the United Nations to help developing countries fight climate change.
This investment is part of the US government's commitment to addressing
climate change and supporting global efforts to move towards a cleaner
and more sustainable economy. The money will be used to support projects
that aim to reduce greenhouse gas emissions, promote renewable energy,
and strengthen the ability of developing countries to deal with the
impacts of climate change. This announcement comes as world leaders
gather in Glasgow, Scotland for the COP26 climate summit, where they are
discussing climate finance and the need for more funding to take action
against climate change.

\hypertarget{spoken-english-7}{%
\section{Spoken English}\label{spoken-english-7}}

The United States has just announced that it will invest a whopping \$3
billion in the Green Climate Fund. This fund, established by the United
Nations, helps developing countries fight climate change. The US
government is making this investment as part of its commitment to tackle
climate change and support global efforts to transition to a cleaner,
greener economy. The money will be used to support projects that aim to
reduce greenhouse gas emissions, promote renewable energy, and help
developing countries become more resilient to the impacts of climate
change. This announcement comes as world leaders gather in Glasgow,
Scotland for the COP26 climate summit, where they are discussing climate
finance and the urgent need for more funding to take action against
climate change.

\hypertarget{spoken-english-with-pause-tag-7}{%
\section{Spoken English with pause
tag}\label{spoken-english-with-pause-tag-7}}

The United States has just announced that it will invest a whopping \$3
billion in the Green Climate Fund. This fund, established by the United
Nations, helps developing countries fight climate change. The US
government is making this investment as part of its commitment to tackle
climate change and support global efforts to transition to a cleaner,
greener economy. The money will be used to support projects that aim to
reduce greenhouse gas emissions, promote renewable energy, and help
developing countries become more resilient to the impacts of climate
change. This announcement comes as world leaders gather in Glasgow,
Scotland for the COP26 climate summit, where they are discussing climate
finance and the urgent need for more funding to take action against
climate change.

\hypertarget{vocabulary-7}{%
\section{Vocabulary}\label{vocabulary-7}}

\begin{itemize}
\tightlist
\item
  Green Climate Fund /ɡriːn ˈklaɪmət fʌnd/: 绿色气候基金
\item
  greenhouse gas emissions /ˈɡriːnhaʊs ɡæs ɪˈmɪʃənz/: 温室气体排放
\item
  renewable energy /rɪˈnjuːəbəl ˈɛnərdʒi/: 可再生能源
\item
  climate resilience /ˈklaɪmət rɪˈzɪliəns/: 气候适应能力
\item
  COP26 climate summit /kɒp twɛnti sɪks ˈklaɪmət ˈsʌmɪt/: COP26气候峰会
\item
  climate finance /ˈklaɪmət ˈfaɪnæns/: 气候融资
\item
  funding /ˈfʌndɪŋ/: 资金支持
\item
  take action against climate change /teɪk ˈækʃən əˈɡɛnst ˈklaɪmət
  ʧeɪndʒ/: 采取行动应对气候变化
\end{itemize}

\bookmarksetup{startatroot}

\hypertarget{us-to-invest-3-billion-in-green-climate-fund-6}{%
\chapter{US to Invest \$3 Billion in Green Climate
Fund}\label{us-to-invest-3-billion-in-green-climate-fund-6}}

\hypertarget{professional-english-8}{%
\section{Professional English}\label{professional-english-8}}

The United States has announced plans to invest \$3 billion into the
Green Climate Fund, according to sources familiar with the matter. The
investment is aimed at supporting global efforts to combat climate
change and promote sustainable development. The Green Climate Fund,
established in 2010, provides financial assistance to developing
countries to help them reduce greenhouse gas emissions and adapt to the
impacts of climate change. The US investment comes as part of the
country's commitment to rejoin the Paris Agreement and take a leading
role in global climate action. This move is expected to encourage other
countries to increase their contributions to the Green Climate Fund and
accelerate the transition to a low-carbon economy.

\hypertarget{simplified-english-8}{%
\section{Simplified English}\label{simplified-english-8}}

The United States has announced that it will invest \$3 billion in the
Green Climate Fund. This fund was created in 2010 to help developing
countries reduce their greenhouse gas emissions and cope with the
effects of climate change. The US investment is part of its commitment
to rejoin the Paris Agreement and play a major role in addressing
climate change. By investing in the Green Climate Fund, the US hopes to
encourage other countries to contribute more and speed up the transition
to a low-carbon economy. This investment will support global efforts to
combat climate change and promote sustainable development.

\hypertarget{spoken-english-8}{%
\section{Spoken English}\label{spoken-english-8}}

The United States has just announced that it will invest a whopping \$3
billion in the Green Climate Fund. This fund was set up back in 2010 to
help developing countries cut down on their greenhouse gas emissions and
deal with the impacts of climate change. Now, the US is jumping on board
and putting its money where its mouth is. By investing in the Green
Climate Fund, the US wants to show the world that it's serious about
tackling climate change. And it's hoping that other countries will
follow suit and chip in more money too. The goal? To speed up the shift
to a low-carbon economy and make a real difference in the fight against
climate change.

\hypertarget{spoken-english-with-pause-tag-8}{%
\section{Spoken English with pause
tag}\label{spoken-english-with-pause-tag-8}}

The United States has just announced that it will invest a whopping \$3
billion in the Green Climate Fund. This fund was set up back in 2010 to
help developing countries cut down on their greenhouse gas emissions and
deal with the impacts of climate change. Now, the US is jumping on board
and putting its money where its mouth is. By investing in the Green
Climate Fund, the US wants to show the world that it's serious about
tackling climate change. And it's hoping that other countries will
follow suit and chip in more money too. The goal? To speed up the shift
to a low-carbon economy and make a real difference in the fight against
climate change.

\hypertarget{vocabulary-8}{%
\section{Vocabulary}\label{vocabulary-8}}

\begin{itemize}
\tightlist
\item
  Green Climate Fund /ɡriːn ˈklaɪmət fʌnd/: 绿色气候基金
\item
  greenhouse gas emissions /ˈɡriːnhaʊs ɡæs ɪˈmɪʃənz/: 温室气体排放
\item
  sustainable development /səˈsteɪnəbəl dɪˈvɛləpmənt/: 可持续发展
\item
  Paris Agreement /ˈpærɪs əˈɡriːmənt/: 巴黎协定
\item
  low-carbon economy /loʊ ˈkɑːrbən ɪˈkɑːnəmi/: 低碳经济
\item
  tackle /ˈtækəl/: 解决,应对
\item
  chip in /tʃɪp ɪn/: 凑钱,共同出资
\item
  make a real difference /meɪk ə riːl ˈdɪfərəns/: 产生真正的影响
\item
  fight against /faɪt əˈɡɛnst/: 对抗,与\ldots 作斗争
\end{itemize}

\bookmarksetup{startatroot}

\hypertarget{us-to-invest-3-billion-in-green-climate-fund-7}{%
\chapter{US to Invest \$3 Billion in Green Climate
Fund}\label{us-to-invest-3-billion-in-green-climate-fund-7}}

\hypertarget{professional-english-9}{%
\section{Professional English}\label{professional-english-9}}

The United States has announced plans to invest \$3 billion in the Green
Climate Fund (GCF), according to sources familiar with the matter. The
GCF is a global fund established to support developing countries in
their efforts to combat climate change and promote sustainable
development. The investment is part of the US government's commitment to
addressing climate change and transitioning to a clean energy economy.
The funds will be used to finance projects and initiatives aimed at
reducing greenhouse gas emissions, adapting to the impacts of climate
change, and promoting renewable energy sources. This announcement comes
as countries around the world are increasing their efforts to tackle
climate change and meet the goals of the Paris Agreement.

\hypertarget{simplified-english-9}{%
\section{Simplified English}\label{simplified-english-9}}

The United States has announced that it will invest \$3 billion in the
Green Climate Fund (GCF). The GCF is a global fund that helps developing
countries fight climate change and promote sustainable development. This
investment is part of the US government's commitment to addressing
climate change and transitioning to clean energy. The money will be used
to support projects that reduce greenhouse gas emissions, adapt to the
impacts of climate change, and promote renewable energy sources. This
announcement comes at a time when countries worldwide are stepping up
their efforts to combat climate change and achieve the goals of the
Paris Agreement.

\hypertarget{spoken-english-9}{%
\section{Spoken English}\label{spoken-english-9}}

The United States has just announced that it will invest a whopping \$3
billion in the Green Climate Fund. This fund is a global initiative that
helps developing countries tackle climate change and promote sustainable
development. The US government is making this investment as part of its
commitment to addressing climate change and transitioning to clean
energy. The money will be used to support projects that aim to reduce
greenhouse gas emissions, adapt to the impacts of climate change, and
promote renewable energy sources. This announcement comes at a time when
countries all over the world are ramping up their efforts to combat
climate change and meet the goals of the Paris Agreement.

\hypertarget{spoken-english-with-pause-tag-9}{%
\section{Spoken English with pause
tag}\label{spoken-english-with-pause-tag-9}}

The United States has just announced that it will invest a whopping \$3
billion in the Green Climate Fund. This fund is a global initiative that
helps developing countries tackle climate change and promote sustainable
development. The US government is making this investment as part of its
commitment to addressing climate change and transitioning to clean
energy. The money will be used to support projects that aim to reduce
greenhouse gas emissions adapt to the impacts of climate change and
promote renewable energy sources. This announcement comes at a time when
countries all over the world are ramping up their efforts to combat
climate change and meet the goals of the Paris Agreement.

\hypertarget{vocabulary-9}{%
\section{Vocabulary}\label{vocabulary-9}}

\begin{itemize}
\tightlist
\item
  Green Climate Fund /ɡriːn ˈklaɪmət fʌnd/: 绿色气候基金
\item
  greenhouse gas emissions /ˈɡriːnhaʊs ɡæs ɪˈmɪʃənz/: 温室气体排放
\item
  sustainable development /səˈsteɪnəbəl dɪˈvɛləpmənt/: 可持续发展
\item
  Paris Agreement /ˈpærɪs əˈɡriːmənt/: 巴黎协定
\item
  tackle climate change /ˈtækəl ˈklaɪmət tʃeɪndʒ/: 应对气候变化
\item
  ramp up efforts /ræmp ʌp ˈɛfərts/: 加大努力
\item
  clean energy /kliːn ˈɛnərdʒi/: 清洁能源
\end{itemize}

\bookmarksetup{startatroot}

\hypertarget{bayern-munich-vs-union-berlin-match-postponed-due-to-snow}{%
\chapter{Bayern Munich vs Union Berlin Match Postponed Due to
Snow}\label{bayern-munich-vs-union-berlin-match-postponed-due-to-snow}}

\hypertarget{professional-english-10}{%
\section{Professional English}\label{professional-english-10}}

The match between Bayern Munich and Union Berlin, scheduled to take
place on December 2, has been postponed due to heavy snowfall. The
decision was made by the German Football Association (DFB) after
considering the safety of the players and spectators. The snowfall has
caused difficult travel conditions and made the pitch unsuitable for
play. The DFB announced that a new date for the match will be determined
in the coming days. Both teams were looking forward to the match, as
Bayern Munich is currently leading the Bundesliga standings and Union
Berlin has been performing well this season. Fans will have to wait for
the rescheduled date to witness this exciting clash.

\hypertarget{simplified-english-10}{%
\section{Simplified English}\label{simplified-english-10}}

The match between Bayern Munich and Union Berlin, which was supposed to
happen on December 2, has been postponed due to heavy snow. The German
Football Association (DFB) decided to postpone the match because of
safety concerns. The snow made it difficult for players and fans to
travel, and the pitch was not suitable for playing. The DFB will
announce a new date for the match soon. Both teams were excited about
the match. Bayern Munich is currently leading the Bundesliga, and Union
Berlin has been doing well this season. Fans will have to wait for the
new date to see this exciting game.

\hypertarget{spoken-english-10}{%
\section{Spoken English}\label{spoken-english-10}}

The Bayern Munich versus Union Berlin match, originally scheduled for
December 2, has been postponed due to heavy snowfall. The German
Football Association (DFB) made the decision to postpone the match in
order to prioritize the safety of the players and the fans. The snow has
caused travel difficulties and has made the playing field unsuitable for
the game. The DFB will announce a new date for the match in the coming
days. Both teams were eagerly anticipating this match-up, with Bayern
Munich currently leading the Bundesliga standings and Union Berlin
performing impressively this season. Fans will have to wait for the
rescheduled date to witness this highly anticipated clash.

\hypertarget{spoken-english-with-pause-tag-10}{%
\section{Spoken English with pause
tag}\label{spoken-english-with-pause-tag-10}}

The Bayern Munich versus Union Berlin match, originally scheduled for
December 2, has been postponed due to heavy snowfall. The German
Football Association (DFB) made the decision to postpone the match in
order to prioritize the safety of the players and the fans. The snow has
caused travel difficulties and has made the playing field unsuitable for
the game. The DFB will announce a new date for the match in the coming
days. Both teams were eagerly anticipating this match-up with Bayern
Munich currently leading the Bundesliga standings and Union Berlin
performing impressively this season. Fans will have to wait for the
rescheduled date to witness this highly anticipated clash.

\hypertarget{vocabulary-10}{%
\section{Vocabulary}\label{vocabulary-10}}

\begin{itemize}
\tightlist
\item
  postponed /pəʊstˈpəʊnd/ : 推迟
\item
  heavy snowfall /ˈhɛvi ˈsnəʊfɔːl/ : 大雪
\item
  German Football Association (DFB) /ˈdʒɜːmən ˈfʊtbɔːl ˌəʊsɔːsiˈeɪʃən/ :
  德国足球协会
\item
  prioritize /praɪˈɔːrɪtaɪz/ : 优先考虑
\item
  playing field /ˈpleɪɪŋ fiːld/ : 比赛场地
\item
  rescheduled /ˌriːˈʃɛdjuːld/ : 重新安排时间
\item
  eagerly /ˈiːɡəli/ : 急切地
\item
  match-up /ˈmætʃʌp/ : 比赛对决
\item
  Bundesliga /ˌbʊndəsˈliːɡə/ : 德甲联赛
\item
  highly anticipated /ˈhaɪli ˈæntɪsɪˌpeɪtɪd/ : 高度期待的
\item
  clash /klæʃ/ : 冲突
\end{itemize}

\bookmarksetup{startatroot}

\hypertarget{us-to-invest-3-billion-in-green-climate-fund-8}{%
\chapter{US to Invest \$3 Billion in Green Climate
Fund}\label{us-to-invest-3-billion-in-green-climate-fund-8}}

\hypertarget{professional-english-11}{%
\section{Professional English}\label{professional-english-11}}

The United States has announced plans to invest \$3 billion into the
Green Climate Fund, according to sources familiar with the matter. The
move is seen as a significant step towards the country's commitment to
combat climate change and support sustainable finance initiatives. The
Green Climate Fund, established by the United Nations, aims to assist
developing countries in their efforts to mitigate and adapt to the
impacts of climate change. The investment from the US will help fund
projects related to renewable energy, climate resilience, and
sustainable development in these countries. This announcement comes as
the US seeks to reestablish its leadership role in global climate action
after the previous administration's withdrawal from the Paris Agreement.

\hypertarget{simplified-english-11}{%
\section{Simplified English}\label{simplified-english-11}}

The United States has announced that it will invest \$3 billion into the
Green Climate Fund. This fund, established by the United Nations, aims
to help developing countries deal with the effects of climate change.
The investment from the US will be used to support projects related to
renewable energy, climate resilience, and sustainable development in
these countries. This move is significant because it shows the US's
commitment to fighting climate change and promoting sustainable finance.
It also marks a shift in the country's approach to global climate
action, as it seeks to regain its leadership role after withdrawing from
the Paris Agreement.

\hypertarget{spoken-english-11}{%
\section{Spoken English}\label{spoken-english-11}}

The United States has just announced that it will invest a whopping \$3
billion into the Green Climate Fund. This fund, which was set up by the
United Nations, is all about helping developing countries tackle the
effects of climate change. The money from the US will be used to support
projects that focus on renewable energy, climate resilience, and
sustainable development in these countries. This is a big deal because
it shows that the US is serious about fighting climate change and
supporting sustainable finance. It's also a sign that the US wants to
take the lead again when it comes to global climate action, especially
after it pulled out of the Paris Agreement.

\hypertarget{spoken-english-with-pause-tag-11}{%
\section{Spoken English with pause
tag}\label{spoken-english-with-pause-tag-11}}

The United States has just announced that it will invest a whopping \$3
billion into the Green Climate Fund. This fund, which was set up by the
United Nations, is all about helping developing countries tackle the
effects of climate change. The money from the US will be used to support
projects that focus on renewable energy, {[}pause{]} climate resilience,
{[}pause{]} and sustainable development in these countries. This is a
big deal because it shows that the US is serious about fighting climate
change and supporting sustainable finance. It's also a sign that the US
wants to take the lead again when it comes to global climate action,
{[}pause{]} especially after it pulled out of the Paris Agreement.

\hypertarget{vocabulary-11}{%
\section{Vocabulary}\label{vocabulary-11}}

\begin{itemize}
\tightlist
\item
  Green Climate Fund /ɡriːn ˈklaɪmət fʌnd/: 绿色气候基金
\item
  combat /ˈkɑːmbæt/: 战斗,对抗
\item
  sustainable finance initiatives /səˈsteɪnəbəl ˈfaɪnæns ɪˈnɪʃətɪvz/:
  可持续金融倡议
\item
  mitigate /ˈmɪtɪɡeɪt/: 缓解,减轻
\item
  adapt /əˈdæpt/: 适应
\item
  renewable energy /rɪˈnuːəbəl ˈɛnərdʒi/: 可再生能源
\item
  climate resilience /ˈklaɪmət rɪˈzɪliəns/: 气候适应能力
\item
  sustainable development /səˈsteɪnəbəl dɪˈvɛləpmənt/: 可持续发展
\item
  withdrawal /wɪðˈdrɔːəl/: 撤回,退出
\item
  leadership role /ˈliːdərʃɪp roʊl/: 领导角色
\item
  tackle /ˈtækəl/: 应对,解决
\item
  whopping /ˈwɑːpɪŋ/: 巨大的,庞大的
\item
  serious about /ˈsɪriəs əˈbaʊt/: 对\ldots 认真,严肃对待
\item
  sign /saɪn/: 迹象,标志
\item
  take the lead /teɪk ðə lid/: 带头,领先
\item
  global climate action /ˈɡloʊbəl ˈklaɪmət ˈækʃən/: 全球气候行动
\item
  pulled out of /pʊld aʊt əv/: 退出,撤出
\item
  Paris Agreement /ˈpærɪs əˈɡriːmənt/: 巴黎协定
\end{itemize}

\bookmarksetup{startatroot}

\hypertarget{us-unveils-plan-to-reduce-climate-super-pollutants-in-oil-and-gas-industry}{%
\chapter{US Unveils Plan to Reduce Climate Super-Pollutants in Oil and
Gas
Industry}\label{us-unveils-plan-to-reduce-climate-super-pollutants-in-oil-and-gas-industry}}

\hypertarget{professional-english-12}{%
\section{Professional English}\label{professional-english-12}}

The United States has announced its plan to tackle climate
super-pollutants in the oil and gas industry ahead of the 28th UN
Climate Change Conference of the Parties (COP 28). The plan aims to
reduce emissions of methane, a potent greenhouse gas, and other
super-pollutants by implementing stricter regulations and promoting the
use of cleaner technologies. The US government plans to work closely
with industry stakeholders to achieve these goals by 2023. This
initiative is part of the country's broader efforts to combat climate
change and transition to a more sustainable energy future.

Methane, which is released during the extraction, production, and
transportation of oil and gas, is estimated to be over 80 times more
potent than carbon dioxide in terms of its warming potential over a
20-year period. By targeting methane emissions, the US aims to
significantly reduce its contribution to global warming. The plan
includes measures such as strengthening leak detection and repair
requirements, improving data collection and reporting, and incentivizing
the adoption of methane reduction technologies. These actions are
expected to not only mitigate climate change but also improve air
quality and public health.

\hypertarget{simplified-english-12}{%
\section{Simplified English}\label{simplified-english-12}}

The United States has revealed its strategy to address the issue of
climate super-pollutants in the oil and gas industry before the 28th UN
Climate Change Conference of the Parties (COP 28). The objective of this
plan is to decrease the release of methane, a potent greenhouse gas, and
other super-pollutants by implementing stricter regulations and
promoting the use of cleaner technologies. The US government intends to
collaborate closely with industry stakeholders to achieve these targets
by 2023. This initiative is part of the country's broader efforts to
combat climate change and transition to a more sustainable energy
future.

Methane, which is emitted during the extraction, production, and
transportation of oil and gas, is estimated to have a warming potential
over a 20-year period that is more than 80 times greater than carbon
dioxide. By focusing on reducing methane emissions, the US aims to
significantly decrease its contribution to global warming. The plan
includes actions such as enhancing requirements for detecting and
repairing leaks, improving data collection and reporting, and providing
incentives for the adoption of technologies that reduce methane. These
measures are expected to not only mitigate climate change but also
enhance air quality and public health.

\hypertarget{spoken-english-12}{%
\section{Spoken English}\label{spoken-english-12}}

The United States has just announced its plan to tackle climate
super-pollutants in the oil and gas industry. This comes ahead of the
28th UN Climate Change Conference of the Parties. The goal is to reduce
emissions of methane, a potent greenhouse gas, and other
super-pollutants. The US government wants to do this by implementing
stricter regulations and promoting the use of cleaner technologies.
They're working closely with industry stakeholders to achieve these
goals by 2023. This is part of their broader efforts to combat climate
change and transition to a more sustainable energy future.

Methane is released during the extraction, production, and
transportation of oil and gas. It's estimated to be over 80 times more
potent than carbon dioxide in terms of its warming potential over a
20-year period. So, by targeting methane emissions, the US aims to
significantly reduce its contribution to global warming. The plan
includes measures like strengthening leak detection and repair
requirements, improving data collection and reporting, and incentivizing
the adoption of methane reduction technologies. These actions will not
only help mitigate climate change but also improve air quality and
public health.

\hypertarget{spoken-english-with-pause-tag-12}{%
\section{Spoken English with pause
tag}\label{spoken-english-with-pause-tag-12}}

The United States has just announced its plan to tackle climate
super-pollutants in the oil and gas industry. This comes ahead of the
28th UN Climate Change Conference of the Parties. The goal is to reduce
emissions of methane, a potent greenhouse gas, and other
super-pollutants. The US government wants to do this by implementing
stricter regulations and promoting the use of cleaner technologies.
They're working closely with industry stakeholders to achieve these
goals by 2023. This is part of their broader efforts to combat climate
change and transition to a more sustainable energy future.

Methane is released during the extraction, production, and
transportation of oil and gas. It's estimated to be over 80 times more
potent than carbon dioxide in terms of its warming potential over a
20-year period. So, by targeting methane emissions the US aims to
significantly reduce its contribution to global warming. The plan
includes measures like strengthening leak detection and repair
requirements improving data collection and reporting and incentivizing
the adoption of methane reduction technologies. These actions will not
only help mitigate climate change but also improve air quality and
public health.

\hypertarget{vocabulary-12}{%
\section{Vocabulary}\label{vocabulary-12}}

\begin{itemize}
\tightlist
\item
  super-pollutants /ˈsuːpər pəˈluːtənts/: 超级污染物
\item
  greenhouse gas /ˈɡriːnhaʊs ɡæs/: 温室气体
\item
  emissions /ɪˈmɪʃənz/: 排放物
\item
  regulations /ˌreɡjʊˈleɪʃənz/: 规定
\item
  sustainable /səˈsteɪnəbl/: 可持续的
\item
  extraction /ɪkˈstrækʃən/: 提取
\item
  potent /ˈpoʊtnt/: 强大的
\item
  warming potential /ˈwɔrmɪŋ pəˈtenʃəl/: 加热潜力
\item
  mitigate /ˈmɪtɪɡeɪt/: 缓解
\item
  incentivize /ɪnˈsɛntɪvaɪz/: 激励
\item
  adoption /əˈdɒpʃən/: 采用
\item
  transition /trænˈzɪʃən/: 过渡
\item
  air quality /ɛr ˈkwɒləti/: 空气质量
\item
  public health /ˈpʌblɪk hɛlθ/: 公共健康
\end{itemize}

\bookmarksetup{startatroot}

\hypertarget{walmart-denies-advertising-on-social-platform-x}{%
\chapter{Walmart denies advertising on social platform
X}\label{walmart-denies-advertising-on-social-platform-x}}

\hypertarget{professional-english-13}{%
\section{Professional English}\label{professional-english-13}}

Walmart, the retail giant, has denied rumors that it will be advertising
on the social platform X. The company clarified that it has no plans to
advertise on the platform in the near future. This statement comes after
speculation arose that Walmart would be partnering with X to promote its
products and services. However, Walmart's spokesperson emphasized that
the company is focused on other advertising strategies and does not have
any agreements or plans with X. Walmart remains committed to its
existing marketing initiatives and will continue to explore new
opportunities to reach its customers.

\hypertarget{simplified-english-13}{%
\section{Simplified English}\label{simplified-english-13}}

Walmart, the large retail company, has stated that it is not going to
advertise on the social platform X. This announcement comes after there
were rumors that Walmart would be working with X to promote its products
and services. However, Walmart has clarified that it has no plans to
advertise on X in the near future. The company is focused on other
advertising strategies and does not have any agreements or plans with X.
Walmart will continue to use its current marketing methods and will also
look for new ways to connect with its customers.

\hypertarget{spoken-english-13}{%
\section{Spoken English}\label{spoken-english-13}}

Walmart, the retail giant, says it's not going to advertise on the
social platform X. There were rumors that Walmart would be partnering
with X to promote its products and services, but the company has denied
those claims. Walmart's spokesperson made it clear that they have no
plans to advertise on X in the near future. Instead, Walmart is focusing
on other advertising strategies and has no agreements or plans with X.
They will continue with their current marketing initiatives and explore
new opportunities to reach their customers.

\hypertarget{spoken-english-with-pause-tag-13}{%
\section{Spoken English with pause
tag}\label{spoken-english-with-pause-tag-13}}

Walmart, the retail giant, says it's not going to advertise on the
social platform X. There were rumors that Walmart would be partnering
with X to promote its products and services, but the company has denied
those claims. Walmart's spokesperson made it clear that they have no
plans to advertise on X in the near future. Instead, Walmart is focusing
on other advertising strategies and has no agreements or plans with X.
They will continue with their current marketing initiatives and explore
new opportunities to reach their customers.

\hypertarget{vocabulary-13}{%
\section{Vocabulary}\label{vocabulary-13}}

\begin{itemize}
\tightlist
\item
  retail giant: a very large company in the retail industry /零售巨头/
\item
  social platform: an online platform where users can interact and share
  content /社交平台/
\item
  rumors: unverified information or stories that are circulating /谣言/
\item
  partnering: working together in a partnership /合作/
\item
  promote: to advertise or make something known to a wider audience
  /推广/
\item
  advertising strategies: methods or plans used to promote products or
  services /广告策略/
\item
  agreements: formal arrangements or contracts /协议/
\item
  marketing initiatives: actions or projects undertaken to promote a
  company or its products /营销举措/
\item
  opportunities: chances or possibilities /机会/
\end{itemize}

\bookmarksetup{startatroot}

\hypertarget{uk-lawmakers-say-britain-has-yet-to-make-case-for-digital-pound}{%
\chapter{UK Lawmakers Say Britain Has Yet to Make Case for Digital
Pound}\label{uk-lawmakers-say-britain-has-yet-to-make-case-for-digital-pound}}

\hypertarget{professional-english-14}{%
\section{Professional English}\label{professional-english-14}}

UK lawmakers have stated that Britain has not yet presented a convincing
case for the introduction of a digital pound. The Treasury Committee, a
group of lawmakers responsible for overseeing the country's financial
policies, expressed concerns about the potential risks and benefits of a
digital currency. They emphasized the need for a thorough analysis of
the economic and social implications before moving forward with any
plans. The committee also highlighted the importance of addressing
issues such as financial stability, privacy, and security. The Bank of
England has been exploring the possibility of a central bank digital
currency (CBDC), but lawmakers believe that more evidence is required to
justify its implementation.

\hypertarget{simplified-english-14}{%
\section{Simplified English}\label{simplified-english-14}}

Lawmakers in the UK have said that the country has not yet provided
enough reasons to support the idea of introducing a digital pound. The
Treasury Committee, a group of lawmakers who oversee the country's
financial policies, has expressed concerns about the possible risks and
benefits of a digital currency. They believe that a detailed analysis of
the economic and social impacts should be conducted before any decisions
are made. The committee also stressed the importance of addressing
issues like financial stability, privacy, and security. The Bank of
England has been looking into the possibility of creating a digital
currency controlled by the central bank, but lawmakers think that more
evidence is needed to justify its implementation.

\hypertarget{spoken-english-14}{%
\section{Spoken English}\label{spoken-english-14}}

Lawmakers in the UK say that Britain hasn't made a strong enough case
for introducing a digital pound. The Treasury Committee, a group of
lawmakers who oversee the country's financial policies, has concerns
about the potential risks and benefits of a digital currency. They
believe that a thorough analysis of the economic and social implications
should be done before moving forward with any plans. The committee also
emphasized the importance of addressing issues like financial stability,
privacy, and security. The Bank of England has been exploring the idea
of a central bank digital currency, but lawmakers think that more
evidence is needed to justify implementing it.

\hypertarget{spoken-english-with-pause-tag-14}{%
\section{Spoken English with pause
tag}\label{spoken-english-with-pause-tag-14}}

Lawmakers in the UK say that Britain hasn't made a strong enough case
for introducing a digital pound. The Treasury Committee, a group of
lawmakers who oversee the country's financial policies, has concerns
about the potential risks and benefits of a digital currency. They
believe that a thorough analysis of the economic and social implications
should be done before moving forward with any plans. The committee also
emphasized the importance of addressing issues like financial stability,
privacy, and security. The Bank of England has been exploring the idea
of a central bank digital currency but lawmakers think that more
evidence is needed to justify implementing it.

\hypertarget{vocabulary-14}{%
\section{Vocabulary}\label{vocabulary-14}}

\begin{itemize}
\tightlist
\item
  digital pound /ˈdɪdʒɪtl paʊnd/: 数字英镑
\item
  Treasury Committee /ˈtrɛʒəri kəˈmɪti/: 财政委员会
\item
  risks and benefits /rɪsks ænd ˈbɛnɪfɪts/: 风险和利益
\item
  thorough analysis /ˈθʌroʊ əˈnæləsɪs/: 彻底分析
\item
  financial stability /faɪˈnænʃəl stəˈbɪləti/: 金融稳定性
\item
  privacy /ˈpraɪvəsi/: 隐私
\item
  security /sɪˈkjʊrəti/: 安全性
\item
  central bank digital currency /ˈsɛntrəl bæŋk ˈdɪdʒɪtl ˈkɜrənsi/:
  央行数字货币
\item
  justify /ˈdʒʌstɪfaɪ/: 证明\ldots 的合理性
\end{itemize}



\end{document}
